\documentclass{article}
\usepackage{graphicx, float, amsmath,mathrsfs, hyperref,color,listings}
 \lstset{breaklines=true} 
\usepackage{amssymb}
\usepackage{upquote} %allows the ' symbol to be copy pasted from verbatim into code windows
\usepackage{setspace}
\usepackage{moreverb}
\usepackage[mathletters]{ucs}
\usepackage[utf8x]{inputenc}
\usepackage[a4paper]{geometry}


\geometry{top=1.0in, bottom=1.0in, left=1.0in, right=1.0in}

\newcommand{\mat}[1]{\begin{bmatrix} #1 \end{bmatrix}}
\newcommand{\htx}[2]{\hspace{ #1 cm} \text{ #2 } }
\newcommand{\es}[1]{\begin{equation*}\begin{split} #1 \end{split} \end{equation*}} % or use es+tab with latex in sublime text 2 with your other snippet
\newcommand{\vt}[1]{\begin{verbatimtab} #1 \end{verbatimtab}}
\newcommand{\xbt}{$\bar{X}_n$} %\xb in text. %note, in the snippet file we need to add extra \ to escape the dollar sign, else it gets confused with dollar-sign-zero below!
\newcommand{\xb}{\bar{X}_n} %\xb in math
\newcommand{\s}{\sigma} %\in math
\newcommand{\ra}{\rightarrow} %\in math
\newcommand{\inv}{^{-1}} %\in math
\newcommand{\sitn}{\sum_{i=1}^n } %in math
\newcommand{\snti}{\sum_{n=1}^\infty}
\newcommand{\sjtn}[1]{\sum_{#1=1}^{n} } %in 
\newcommand{\sntk}[1]{\sum_{n=1}^{#1} } %in math
\newcommand{\siid}{\sim_{iid} } %in math
\newcommand{\lr}[1]{\left( #1 \right) } %in math


\newcommand{\interAdapt}{\textsf{interAdapt }}

\title{\interAdapt – An Interactive Tool for Designing and Evaluating Randomized Trials with Adaptive Enrollment Criteria}
\date{}
\author{Aaron Fisher, Michael Rosenblum and Harris Jaffee}

\begin{document}
\maketitle

%Notes for Michael
%Changed all the notation to to match, except for p_01 and p_10, which I left as p_0t and p_1c, etc.
%It looks like there's some checks to ensure that the range on the x axis of the plots always includes zero? I wasn't sure, so I just formally added this contraint to the min/max arguments on the upper/lower bounds in boxTable.csv
%SS now refers to the standard design in the subpop, NOT the sample size
%changed the word "fixed" to "standard" in Design.R plot/table labels, but did not change the variable names from _fixed_ to _standard_ as I worried this might cause us merge conflicts. In the future, it would might be a clean idea to search and replace them. To the user it won't matter of course.
%Need acknowledgements section talking about the grant



% Our adaptive designs have a prespecified parameter $k^*$ such that
% only subpopulation 1 may be enrolled during each stage $k>k^*$.
% For each $k>k^*$, if the trial continues after stage $k-1$, then $n_k$
% participants from subpopulation 1 are enrolled during stage $k$.
% If enrollment is restricted to subpopulation 1 at the end of  some
% stage $k' < k^*$, then for
% each subsequent stage $k$ before the trial is stopped: if $k \leq
% k^*$, $\pi_1 n_k$ participants from subpopulation 1 are enrolled; if
% $k>k^*$ (i.e., it was preplanned that only subpopulation 1 could be
% enrolled during stage $k$), then $n_k$ participants from subpopulation
% 1 are enrolled in stage $k$.



% Also a few other minor things to do:
% *Final error check on the software, once it's all put together. 
% *Once we have the final draft of the paper, we need to update the Rmd files that generate the knitr reports. Then we can put fixed versions of the R files on the Spark server.
% *We also may want to create a different repository on github, one that doesn't have all our extra comments and history in it. Or maybe not, I'm not sure what the convention is there.
%Because it's possibly to cross both an efficacy boundary and a futility boundary in the same step, we always check for efficacy first?


%Sync your notation to match his in the paper.
%ALSO NEED TO SYNC UP NOTATION IN \interAdapt!

%Notes on notation (From AF)
%I changed A_ik to T_ik (treatment indicator)
%Also, changed p_10 to p_1c
%Otherwise it should be the same notation as the MISTIE paper.
%Standard designs are abbreviated as SS and SC
	%(Need to update labels in \interAdapt stuff. Design.R, boxtable,slider table, and knitr.)
%As you suggested the proportionality constants are now in the form of, for example, f_{SC} (for futility constant for standard design in the combined population,), e_{AD,1} (efficacy proportionality constant for subpop $1$ in the adaptive design. 
%a_C is the proportion of α allocated for HOC.

%!!! Do we need to cite Shiny?

%Notes on converting this file to the shiny Rmd report file.
% When copy pasting to Rmd, you need to take out \lr{}, \htx, \item, \subitem, * → \star. Sections to #, chenge ``quoted item'' to ''quoted item''
%in multi line eq's, can't start a line with "-" (minus sign). Also, I don't think we can do multi-line equations... Now I have them stuck on one line.
%Comments will no longer be commented out!
%Double check citations to make sure they work and that bib files are the same.
\begin{abstract}

We consider the problem of designing a randomized trial when there is prior evidence that the experimental treatment may be more effective for certain groups of patients, such as those with a certain biomarker or risk score at baseline. Randomized trial designs have been proposed that dynamically adapt enrollment criteria to target precisely those benefiting from treatment. Such trial designs aim to learn if the treatment benefits the overall population, only a certain subpopulation, or neither.
%It can be challenging to construct such designs, and to evaluate their performance compared to standard designs. To address this challenge,
We introduce the \interAdapt software tool, which provides a user friendly interface for constructing and evaluating certain adaptive trial designs. These designs are automatically compared to standard (non-adaptive) designs in terms of the following performance criteria:  power, sample size, and trial duration.
Unlike existing software, \interAdapt is open-source and cross-platform, and is the first to implement the group sequential, adaptive enrichment designs of \cite{Rosenblum2013AdaptMISTIE}. %!!?? Update this bit about how it relates to ?

\end{abstract}


\section*{1. Introduction}
Group sequential, randomized trial designs involve rules for early stopping based on analyses of accrued data. Such early stopping could occur if there is strong evidence early in the trial of benefits or harms of the new treatment being studied. Adaptive enrichment designs expand on group sequential designs but including rules for changing enrollment criteria based on preliminary data accrued in the ongoing trial.
Here, we introduce the \interAdapt software tool, which provides a user-friend interface for exploring certain types of adaptive enrichment designs, and for comparing these designs against standard, group sequential designs with non-adaptive enrollment criteria. The software can either be run locally, as an R package, or accessed online through a web browser.

\interAdapt specifically considers the class of designs presented by \cite{Rosenblum2013AdaptMISTIE}, which
we refer to simply as ``adaptive designs." The software compares the performance of these designs to ``standard designs,'' defined to  be group sequential designs where the enrollment criteria cannot be changed during the trial (but the trial may be stopped early). 

\interAdapt is designed to be used by clinical investigators, and provides information that can help users quickly determine if an adaptive design offers a tangible improvement over standard designs in the context of their specific trial. All calculations are based on simulations, which run in less than 1 minute on a standard commercial laptop. Several user inputs are available to allow the user to describe the context of their trial. Alternatively, users can upload data from previous studies, and \interAdapt will automatically compute the relevant parameters for the current trial. Once entered, the full set of input parameters can be saved to the user's computer for use in future sessions. Results of the design comparisons can be immediately downloaded in the form of either csv-tables, or printable, html-based reports.


To demonstrate our designs and software, we consider the problem of 
%The motivating example for our work comes from the
 planning a Phase III trial for a new surgical treatment of stroke, which is considered by (Rosenblum et al. 2013)\cite{Rosenblum2013AdaptMISTIE}.
The new treatment is known as Minimally-Invasive Surgery Plus rt-PA for Intracerebral Hemorrhage (MISTIE), and is described in more detail in (Morgan 2008)\cite{MISTIE_prelim2008}. Leading up to this trial, there was more prior data on the efficacy of the treatment for patients with little or no intraventricular hemorrhage (IVH) at baseline (referred to as small IVH patients) than for patients with large IVH volume. The goal of the trial was to determine whether MISTIE is effective for the combined population of those with small or large IVH, and, if not, to determine whether MISTIE is effective for the small IVH population. A standard trial design, e.g., one enrolling the combined population throughout the trial, or enrolling only small IVH participants throughout the trial, may be inefficient at simultaneously answering these questions.
An alternative is to use an adaptive trial design, which would first recruit from the combined population, and then decide whether to restrict enrollment based on results from interim  analyses. %Comparable standard designs would either enroll all , or would restrict to enrolling only patients with small IVH.
Though we focus on this stroke trial application throughout, our software tool can be applied in many disease areas.

%Need to use \textsf{interAdapt} so that the period isn't an extra space away.
In Section 2, we  formally define the hypothesis testing problem to be addressed different trial designs. In Section 3, we compare our software to the most similar, currently available commercial software, AptivSolutions ADDPLAN PE (Patient Enrichment). 
In Section 4, we describe how to install \interAdapt on a personal computer, and how to access it online through a web browser. Section 5 describes the inputs available when using \textsf{interAdapt}, and discusses the interpretation of the application's output. In Section 6, we present an example demonstrating how an adaptive design is  created and analyzed with \textsf{interAdapt}. 

\section*{2. Problem Description}

Consider the case where we have two subpopulations, referred to as subpopulation $1$ and subpopulation $2$. These must be specified before the trial starts, and be defined in terms of participant attributes measured at baseline (e.g., having a high initial severity of disease or a certain biomarker value). 
We focus on situations where  there is suggestive, prior evidence that the treatment may be more likely to benefit subpopulation $1$.
%though it is possible that the treatment will benefit both subpopulations.
Let $π_1$ and $π_2$ denote the proportion of patients in subpopulations 1 and 2, respectively. 

Both the adaptive and standard designs discussed here consist of ongoing enrollment, and include predetermined rules for stopping the trial early based on interim analyses of currently enrolled patients. We discretize each trial into $K$ stages, and say that the $k ^{th}$ stage will be completed once a certain number of additional patients have been enrolled. The number of pateints required to end the $k^{th}$ stage depends on the trial design, and may also depend on deterministic decision rules applied to the data acrued in previous stages. In stages when both subpopulations are being recruited, we assume that the proportion of newly recruited patients who are in subpopulation 1 is equal to $π_1$. An interim analysis is done at the end of each stage, which may lead us to stop the trial early if there is either strong evidence of treatment efficacy, or strong evidence of treatment futility.

Let $Y_{i,k}$ be the a binary outcome variable for the $i^{th}$ patient recruited in stage $k$, where $Y_{i,k}=1$ indicates a successful outcome. Let $T_{i,k}$ be an indicator of the event that the $i^{th}$ patient recruited in stage $k$ is assigned to the treatment. \interAdapt assumes equal probability of being assigned to either treatment or control.

For subpopulation $1$, denote the probability of a successful outcome under treatment as $p_{1t}$, and the probability of a successful outcome under control as $p_{1c}$. Similarly for population $2$, let $p_{2t}$ denote the probability of a success under treatment, and $p_{2c}$ denote the probability of a success under control. We define the average treatment effect for a given population as difference in the probability of a successful outcome between the treatment and control groups.


In the remainder of this section we give an overview of the relevant concepts needed to understand and use \textsf{interAdapt}. A more detailed discussion of the theoretical context, and of the parameter calculation procedure, can be found in (Rosenblum et al. 2013)\cite{Rosenblum2013AdaptMISTIE}.
 
\subsection*{Hypotheses}

We focus on testing the null hypothesis of no treatment effect in subpopulation $1$, and the null hypothesis of no treatment effect in the combined population. The two hypotheses are defined respectively as

\begin{itemize}
\item $H_{01}$: $p_{1t}-p_{1c}≤0$;%The treatment effect in subpopulation $1$ is less than or equal to zero.
\item $H_{0C}$: $π_1(p_{1t}-p_{1c}) + π_2(p_{2t}-p_{2c}) ≤ 0$. %The average treatment effect in the combined population is less than or equal to zero.	
\end{itemize}

\interAdapt shows the decision rules for an adaptive design testing both of these hypotheses. The application compares this adaptive design to the standard design testing only $H_{0C}$, and to the standard design testing only $H_{01}$. We refer to the adaptive design as $AD$, and refer to the two standard designs as $SC$ and $SS$ respectively. All three trials contain $K$ stages, and the decision to entirely stop the trial early can be made at the end of any stage. The trials differ in that $SC$ and $SS$ never change their enrollment criteria, while $AD$ may switch to enroll only patients from subpopulation $1$.

Note that the designs discussed here are not the same as the standard designs discussed in section 6.1 of (Rosenblum et al. 2013)\cite{Rosenblum2013AdaptMISTIE}, which test both hypothesis simultaneously. Implementing standard designs such as those discussed in (Rosenblum et al. 2013)\cite{Rosenblum2013AdaptMISTIE} into the software is an area of future research.

Whenever any of the trials $AD$, $SC$ or $SS$ is stopped early, there will be some patients who have been enrolled but who’s outcomes have not yet been measured. These patients are referred to as “overrunning” or “pipeline” patients. \interAdapt currently discards measurements from these overruning patients in the final analysis. Incorporating these measurements is a goal for future work.
%AF note to self: Michael has mentioned that there are a few ways to do this. Jason does a bayesian prediction of passing a threshold once the patients come in. There are also ways to set "double thresholds," one threshold for when to stop, and one (usually lower) threshold for whether or not to declare significant results once the overrunning patients are measured.



\subsection*{Test Statistics}

We calculate three z-statistics at the end of each stage, one for the treatment effect in the combined population, and one for the treatment effect in each subpopulation. 

Denote $Z_{C,k}$ as the z-statistic for the estimated treatment effect in the combined population, which is based on the data from all patients with outcomes recorded by the end of stage $k$. The test statistic $Z_{C,k}$ takes the following form:

\begin{equation*}\begin{split}
Z_{C,k}&=\left[
\frac{\sum_{k'=1}^k \sum_{i=1}^{n_{k'}}Y_{i,k'}T_{i,k'} }
{\sum_{k'=1}^k \sum_{i=1}^{n_{k'}}T_{i,k}}
-
\frac{\sum_{k'=1}^k \sum_{i=1}^{n_{k'}} Y_{i,k'}(1-T_{i,k'})} 
{\sum_{k'=1}^k \sum_{i=1}^{n_{k'}}(1-T_{i,k})}
\right] \\
& \htx{1}{} \times
\left\lbrace
\lr{     \frac{2}{  \sum_{k'=1}^{k} n_{k'}  }       }
\lr{
\sum_{s ∈ \{ 1,2\}} π_s[p_{sc}(1-p_{sc}) + p_{st}(1-p_{st})]
}
\right\rbrace ^{-1/2}
\end{split}\end{equation*}


The term in square brackets is the difference in sample means between the treatment and control groups. The term in curly braces is the variance of this difference in sample means.

Let $Z_{1,k}$ and $Z_{2,k}$ denote the analogous test statistics for the z-statistics of the estimated treatment effect in subpopulations $1$ and $2$. The explicit form of $Z_{1,k}$ can be written as follows, where $A_{i,k}$ is the indicator that the $i ^{th}$ subject recruited in stage $k$ is a member of subpopulation $1$:


\begin{equation*}\begin{split}
Z_{1,k}&=\left[
\frac{\sum_{k'=1}^k \sum_{i=1}^{n_{k'}}Y_{i,k'}T_{i,k'}A_{i,k'} }
{\sum_{k'=1}^k \sum_{i=1}^{n_{k'}}T_{i,k}A_{i,k'}}
-
\frac{\sum_{k'=1}^k \sum_{i=1}^{n_{k'}} Y_{i,k'}(1-T_{i,k'})A_{i,k'}} 
{\sum_{k'=1}^k \sum_{i=1}^{n_{k'}}(1-T_{i,k})A_{i,k'}}
\right] \\
& \htx{1}{} \times
\left\lbrace
\lr{     \frac{2}{\sum_{k'=1}^k \sum_{i=1}^{n_{k'}}A_{i,k'}}       }
\lr{
π_1[p_{1c}(1-p_{1c}) + p_{1t}(1-p_{1t})]
}
\right\rbrace ^{-1/2}
\end{split}\end{equation*}

We can write $Z_{2,k}$ in an analagous form using an indicator of memebership in population $2$.

The decision rules defined later on in this section for testing $H_{0C}$ and $H_{01}$ consist of critical boundaries for $(Z_{C,1},Z_{C,2},...Z_{C,K})$, $(Z_{1,1},Z_{1,2},...Z_{1,K})$, and $(Z_{2,1},Z_{2,2},...Z_{2,K})$. To calculate the familywise Type I error of any given set of decision rules, we make use of the multivariate distribution of $(Z_{C,1},Z_{C,2},... Z_{C,K}, Z_{1,1},Z_{1,2},...Z_{1,K})$, which can be shown to be normal with a known covariance matrix (Jennison and Turnbull, 1999, Chapter 3)\cite{JennisonTurnbullBook}.



\subsection*{Familywise Type I Error}

In the context of our hypotheses, the Familywise Type I error rate refers to the combined rate of false positives from testing either $H_{0C}$ and $H_{01}$. We say that the Familywise Type I error rate is controlled at level $α$ when the probability of rejecting at least one true hypothesis is less than $α$.

For all three designs, $AD$, $SC$, and $SS$, we require the familywise Type I error rate to be $α$. Since the two standard designs $SS$ and $SC$ each only test a single hypothesis, their familywise Type I error rates equal the Type I error rates of their respective hypothesis tests. We discuss the control of the familywise Type I error rate for the $AD$ design in the next section.

%% A more detailed verion of this paragraph from a previous draft, probably better as it is now:
%%Note that since $SC$ and $SS$ do not test the same hypotheses as $AD$, they are not directly comparable. It is possible to combine $SC$ and $SS$ in order to test both $H_{0C}$ and $H_{01}$, but the trials cannot be done simultaneously without doubling the recruitment rate. Instead, the trials could be combined in sequence by first conducting $SC$, and then conducting $SS$ only if $SC$ does not find a significant effect. This sequence however allow for twice the maximum number of stages (2K). A multiple hypothesis testing correction would also have to be employed in any combination of $SC$ and $SS$. A more detailed discussion of how $SC$ and $SS$ can be combined, and how these combinations compare against $AD$, is given in (MISTIE PAPER). 



\subsection*{Decision Rules for Stopping}

The $SC$ decision rules consist of efficacy and futility boundaries for $H_{0C}$. At each stage $k$, we calculate the test statistic $Z_{C,k}$. If $Z_{C,k}$ is above the efficacy boundary for stage $k$, we reject $H_{0C}$ and end the trial. If $Z_{C,k}$ is between the efficacy and futility boundaries for stage $k$, we continue the trial. If $Z_{C,k}$ is below the futility boundary for stage $k$, we end the trial with the conclusion that we have failed to reject $H_{0C}$. \interAdapt makes the simplification that the number of patients enrolled in each stage of $SC$ is constant ($n_{SC}$), and allows the user to input this per-stage enrollment rate.

The efficacy boundaries for $SC$ are set to be proportional to those described by Wang and Tsiatis (1987). This means that the efficacy boundary for the $k^{th}$ stage is set to $e_{SC}\{(\sum_{k'=1}^{K} n_{k'})/n_k\}^{-δ}$, where $K$ is the total number of stages, $δ$ is a constant in the range $[-.5,.5]$, and $e_{SC}$ is the constant calibrated to ensure the desired familywise Type I error rate. In order to calculate $e_{SC}$, we make use of the fact that the random vector of test statistics ($Z_{C,1},Z_{C,2},…Z_{C,K}$) follows a multivariate normal distribution with a known covariance structure (Jennison and Turnbull, 1999, Chapter 3)\cite{JennisonTurnbullBook}.
Using the ``mvtnorm'' package \cite{mvtnorm} in R to evaluate the multivariate normal distribution function, we find the proportionality constant $e_{SC}$ such that the null probability of $Z_{C,k}$ exceeding $e_{SC}\{(\sum_{k'=1}^{K} n_{k'})/n_k\}^{-δ}$ at any stage $k$ is less than or equal to $α$.

In $SC$, as well as in $SS$ and $AD$, we make use of non-binding futility constants. All three designs are calibrated such that familywise Type I error rate is controlled at level α regardless of whether the futility boundaries are ignored. In calculating power however, we do assume that the futility boundaries are adhered to.

Futility boundaries for the first $K-1$ stages of $SC$ are proportional to $\{(\sum_{k'=1}^{K} n_{k'})/n_k\}^{-δ}$, but with a different proportionality constant, $f_{SC}$. The constant $f_{SC}$ is negative by default, though this is not required. In the $K ^{th}$ stage of the trial, \interAdapt sets the futility bound to be equal to the efficacy bound. This ensures that $Z_{C,K}$ eventually crosses either the efficacy bound or less futility bound.

The decision boundaries for $Z_{1,k}$ in the $SS$ design are defined by exactly the same form. The efficacy boundary for the $k^{th}$ stage is set equal to $e_{SS}\{(\sum_{k'=1}^{K} n_{k'})/n_k\}^{-δ}$, where $e_{SS}$ is the constant that ensures the appropriate Type I error rate. The first $K-1$ futility boundaries for $H_{01}$ are set equal to $f_{SS}\{(\sum_{k'=1}^{K} n_{k'})/n_k\}^{-δ}$,  where $f_{SS}$ is a constant that can be set by the user. The futility boundary in stage $K$ is set equal to the final efficacy boundary in stage $K$. The user can specify the number of patients to enroll in each stage ($n_{SS}$). This per-stage enrollment rate is set to be constant across stages.

Decision boundaries for $AD$ vary from those of the standard designs two ways. First, because $AD$ simultaneously tests $H_{0C}$ and $H_{01}$ it has two sets of decision boundaries. For the $k^{th}$ stage of $AD$, let $u_{C,k}$ and $u_{1,k}$ denote the efficacy boundaries for $H_{0C}$ and $H_{01}$ respectively. The boundaries $u_{C,k}$ and $u_{1,k}$ are set equal to $e_{AD,C}\{(\sum_{k'=1}^{K} n_{k'})/n_k\}^{-δ}$ and $e_{AD,1}\{(\sum_{k'=1}^{K} n_{k'})/n_k\}^{-δ}$ respectively, where $e_{AD,C}$  and $e_{AD,1}$ are constants set such that the probability of rejecting either hypothesis under the global null hypothesis is zero. 
%There are several pairs of $e_{AD,C}$  and $e_{AD,1}$ that satisfy this condition for the Type I error rate. For each possible $ e_{AD,C}$ there is a one to one correspondence with a $e_{AD,1}$, which ensures that the probability of rejecting either hypothesis under the global null is zero. 

The boundaries for stopping the $AD$ design without rejecting the null hypotheses are denoted as $l_{1,k}$ and $l_{2,k}$. These stopping boundaries are defined relative to the test statistics $Z_{1,k}$ and $Z_{2,k}$. The boundaries are set equal to $f_{AD,C}\{(\sum_{k'=1}^{K} n_{k'})/n_k\}^{-δ}$ and $f_{AD,S}\{(\sum_{k'=1}^{K} n_{k'})/n_k\}^{-δ}$ respectively, where $f_{AD,C}$ and $f_{AD,S}$ can be set by the user.

The second way that the decision boundaries of $AD$ differ from those of the standard designs is that we allow more flexibility in the futility boundaries. In each stage $k$, our adaptive design has the option of stopping enrollment in subpopulation 2, based on the treatment effect estimate $Z_{2,k}$. \interAdapt also allows the user to specify a final stage for testing an effect in the total population, denoted by stage $k^*$. Regardless of the results at stage $k^*$, we always stop enrolling from subpopulation $2$ at the end stage $k^*$, if we have not done so already. The futility boundaries $l_{2,k}$ are not defined for $k>k^*$.


For the $AD$ design, the user can specify two stage specific sample sizes, one for stages when both populations are enrolled ($k≤k^*$), and one for stages where only patients in subpopulation $1$ are enrolled ($k>k^*$). We refer to these two sample sizes as $n_1^*$ and $n_k^*$ respectively.
%!!! What are sample sizes for SC AND SS??

As described in (Rosenblum et al. 2013)\cite{Rosenblum2013AdaptMISTIE}, our decision rules in $AD$ consist of the following steps for each stage $k$:

%New rules
\begin{description}
\item 1. (Assess Efficacy) If $Z_{C,k} > u_{C,k}$, reject $H_{0C}$. If $Z_{1,k}>u_{1,k}$, reject $H_{01}$. If either, or both null hypothesis are rejected, stop all enrollment and end the trial.
\item 2. (Assess Futility of the entire trial) Else, if $Z_{1,k} ≤ l_{1,k}$ or if this is the final stage of the trial, stop all enrollment and end the trial for futility, failing to reject either $H_{0C}$ or $H_{01}$.
\item 3. (Assess Futility for $H_{0C}$) Else, if $Z_{2,k} ≤ l_{2,k}$, or if $k\geq k^*$, stop enrollment from subpopulation $2$ in all future stages. In this case, the following steps must then be done:
%(In your draft you say "stop enrolling in all future stages including current one"? What does that mean??)
	\subitem  3.a If $Z_{1,k} > u_{1,k}$, reject $H_{01}$ and stop all enrollment.
	\subitem  3.b If $Z_{1,k} ≤ l_{1,k}$ or if this is the final stage of the trial, conclude that we've fail to reject either $H_{0C}$ or $H_{01}$, and stop all enrollment.
	\subitem  3.c Else, continue to enroll patients from subpopulation $1$. If $k<k^*$ then $π_1n_1^*$ patients should be enrolled in the next stage. If $k≥k^*$, then $n_k^*$ patients should be enrolled in the next stage. For all future stages, ignore steps (1-2), and proceed directly to steps (3.a-3.c).
\item  4. (Continue Enrollment from Combined Population) Else, continue by enrolling $\pi_1 n_1^*$ patients from subpopulation 1 and $\pi_2 n_1^*$ patients from subpopulation 2 for the next stage. Then return to step 1.
\end{description}

The decision rules outputted by \interAdapt represent the feature that enrollment of subpopulation 2 cannot continue after stage $k^*$ by setting the futility boundary $l_{2,k^*}$ equal infinity. This ensures that $Z_{2,k^*}<l_{2,k^*}$. 

To correctly calibrate $e_{AD,C}$  and $e_{AD,1}$, \interAdapt first chooses $e_{AD,C}$ such the probability of falsely rejecting $H_{0C}$ is $a_c α$, where $a_c$ is a fraction between 0 and 1 that can be specified by the user. Then, conditional on $e_{AD,C}$, \interAdapt finds the smallest constant $e_{AD,1}$ such that, under the global null of no treatment effect in either subpopulation, we have %the probability of rejecting either $H_{01}$ or $H_{0C}$ at any stage is less than or equal to $α$. %via a binary search.

\begin{equation*}\begin{split}
P\left(
Z_{C,k}>e_{AD,C} \left\{\frac{\sum_{k'=1}^{K} n_{k'}}{n_k}\right\}^{-δ} \text{  or  } Z_{1,k}> e_{AD,1}\left\{\frac{\sum_{k'=1}^{K} n_{k'}}{n_k}\right\}^{-δ}\text{  for any $k$}
\right) ≤ α 
\end{split}\end{equation*}

The fact that familywise Type I error rate is controled under the global null implies that it is also strongly controled under all hypotheses (Rosenblum et al. 2013)\cite{Rosenblum2013AdaptMISTIE}. 

%OlD RULES
% \begin{description}
% \item (1) Assess Efficacy in the Combined Population: If $Z_{C,k} > u_{C,k}$, reject $H_{0C}$ and stop all enrollment. If $Z_{1,k}>u_{1,k}$, reject $H_{01}$ as well. 
% \item (2) Assess Futility in the Combined Population: Else, if $Z_{C,k} ≤ l_{C,k}$, stop enrolling from subpopulation $2$ in all future stages. In this case when $Z_{1,k}>l_{1,k}$, the following additional steps must be done:
% %(stop enrolling in all future stages including current one? What does that mean?)
% 	\subitem  (a) If $Z_{1,k} > u_{1,k}$, we reject $H_{01}$ and stop all enrollment.
% 	\subitem  (b) If $Z_{1,k} ≤ l_{1,k}$, we fail to reject either $H_{0C}$ or $H_{01}$, and stop all enrollment.
% 	\subitem  (c) Else, we continue to enroll from subpopulation $1$, and re-evaluate steps (2)-(3) at the end of the next stage. 
% \item  (3) If $l_{C,k} < Z_{C,k} ≤ u_{C,k}$, continue enrolling from both subpopulations.
% \end{description}



\section*{3. Related Software}
%Say that our design is an example of enrichment. 

The most comparable available software is AptivSolutions ADDPLAN PE (Patient Enrichment), an impressive, commercial software that implements certain types of adaptive enrichment designs. It has many features that our software does not have. Conversely, there are features of our software that ADDPLAN PE does not have. First, ADDPLAN PE does not implement the class of designs from (Rosenblum et al. 2013)\cite{Rosenblum2013AdaptMISTIE}. Second, in ADDPLAN PE, the user must a priori designate a particular stage (e.g., stage 2) at which a change to enrollment may be made, even though there may be large a priori uncertainty as to when sufficient information will have accrued to make such a decision. In contrast, our software is more flexible, in that one can select designs (by setting $k^*$ to the maximum number of stages) in which the decision to change enrollment criteria can be made at any stage.  

\interAdapt also has the benefits of being cross-platform and open-source, while ADDPLAN PE is commercial software that is only compatible with the Windows OS.

\section*{4. Running \interAdapt}
%Note to mention: Harris edited out a lot of the extra detail about how Shiny works, and some of the redundancy between subsections on that topic.

\interAdapt is an interactive application built on the ``Shiny'' package for the R programming language (\url{http://www.r-project.org/}). The user interface is shown in the user’s default web browser, while the back-end calculations are all done in R. Users can run \interAdapt either by installing R and the \interAdapt R package locally on their computer, or by simply using a web browser view \interAdapt online. Both options are free and quick to set up. However, because online application will slow down noticeably when accessed by multiple users, we encourage heavy users to install \interAdapt locally.


\subsection*{Running \interAdapt  Over the Web}

\interAdapt is currently hosted on the RStudio webserver, and can be accessed simply visiting the link below.
\url{http://spark.rstudio.com/mrosenblum/interAdapt}
%!!! need to kee this link updated!

\subsection*{Running \interAdapt Locally}



To run \interAdapt locally, one must first install the R programming language. R runs on both Windows \& MacOS, with the most current versions available for download at (\url{http://www.r-project.org/}). After downloading and installing R, activating the R application will open an ``R Console'' window where typed commands are executed by R. interAdapt is available as a package for R, and can be installed by typing the lines below into the R Console, while connected to the Internet. The return key must be pressed after each line of code. The first and third lines will cause R to give feedback on the installation progress, which we do not show here.

\vspace{5 mm}
\begin{verbatim}
install.packages('devtools')
library('devtools')
install_github(username='aaronjfisher',repo='interAdapt',subdir='r_package')
\end{verbatim}
\vspace{5 mm}

Once interAdapt has been installed, the application can be run without an internet connection by the opening the R Console and typing.

\vspace{5 mm}
\begin{verbatim}
library('interAdapt')
runInterAdapt()
\end{verbatim}
\vspace{5 mm}




\section*{5. User Interface}

Inputs to \interAdapt can be entered in the side panel on the left, with outputs are shown in the main panel on the right. %Could consider adding a figure, but it's probably fine.
The parameters in the input panel let the user describe known or assumed characteristics of their populations of interest, as well as their trial design parameters. Input parameters include the proportion of patients in each subpopulation, the patient recruitment rate in each subpopulation, and the desired familywise Type I error rate. The output section displays the decision boundaries and trial designs that will satisfy the requirements specified by the user. It also compares the performance of the three designs, $AD$, $SC$ and $SS$. Performance is compared in terms of power, expected sample size, expected trial duration, and expected number of overrunning patients.

All tables generated by \interAdapt can be downloaded as csv files by clicking on the download button beneath the table. Users can also download an automated report of the results by clicking the ``Generate Report'' button at the bottom of the output panel. This report is generated with the ``knitr'' package for R \cite{knitr}. Citations in the report are created using the ``knitcitations'' package \cite{knitcitations}.

\subsection*{Inputs}

Parameters in the input panel are organized into two sections, basic parameters and advanced parameters. To view the different sets of parameters, click the drop down menu titled “Show basic parameters.” 

Basic parameters can be entered using either “Batch mode” or “Interactive mode”. In Batch mode, \interAdapt will not analyze the entered parameters until the “Apply” button is pressed. This allows for several parameters to be changed at once without waiting for \interAdapt to recalculate the results after each individual change. In Interactive mode, \interAdapt will automatically recalculate the results after each change, allowing the user to quickly see the effect of changing one specific input parameter. Switching between Batch mode and Interactive mode can be done using the dropdown menu at the top of the Basic Parameters section. Interactive mode is not available when entering advanced parameters.

To save the current set of inputs, select the dropdown menu titled “Show basic parameters” and select “Show All Parameters and Save/Load Option". From here, you can save the current parameters as a csv file, or load a previously saved csv file of inputs. Regardless of whether \interAdapt is being run online or locally, these saved csv files are always stored on the user's computer. You may also load a 3-column dataset into \interAdapt in the form of a csv, where each row contains information about a patient in the trial. The first column must contain binary indicators of subpopulation, where 1 denotes subpopulation 1, and 2 denotes subpopulation 2. The second column must contain an indicator of the treatment arm ($T_i$), and the third column must contain the binary outcome measurement ($Y_i$). The first row of this dataset file is expected to be a header row of labels, rather than values for the first individual. From this dataset, \interAdapt will calculate $π_1$, $p_{1c}$, $p_{1t}$, $p_{2c}$, and $p_{2t}$, and adjust the input sliders accordingly.

A detailed explanation of each input is given below.

\subsection*{Basic Parameters}
\begin{itemize} 

\item Subpopulation $1$ proportion ($π_1$): The proportion of the population in subpopulation $1$. This is the subpopulation in which we have prior evidence of a stronger treatment effect. 

\item Probability outcome = 1 under control, subpopulation $1$ ($p_{1c}$): The probability of experiencing a successful outcome for control patients in subpopulation $1$. This is used in estimating power and expected sample size of each design.

\item Probability outcome = 1 under control, subpopulation $2$ ($p_{2c}$): The probability of experiencing a successful outcome for control patients in subpopulation $2$. This is used in estimating power and expected sample size of each design.

\item Probability outcome = 1 under treatment for sub-population $1$ ($p_{1t}$): The probability of experiencing a successful outcome for treated patients in subpopulation $1$. Note that a specific effect size is not specified for subpopulation $2$. Instead, \interAdapt generates the relevant performance metrics for a range of several possible effect sizes in subpopulation $2$. This range can be specified in the Advanced Parameters section.%!!!!! wrong!

\item Per stage sample size, combined population ($n_1^*$): The number of patients enrolled in stages 1 through $k^*$ of $AD$. Per stage enrollment for $SC$ and $SS$ can be entered in the advanced parameters section.


\item Per stage sample size for stages where only sub-population 1 is enrolled ($n_k^*$): The number of patients required for each stage after stage $k^*$, in the $AD$ design.

\item Alpha (FWER) Requirement ($α$): The familywise Type I error rate for all hypotheses in the trial. In $AD$, this is the probability of falsely rejecting either $H_{0C}$ or $H_{01}$. In $SC$ it is the probability of falsely rejecting $H_{0C}$. In $SS$ it is the probability of falsely rejecting $H_{01}$.

\item Proportion of Alpha allocated to H0C ($a_C$): To control the familywise Type I error rate in the $AD$ design, the test of $H_{0C}$ is first calibrated to have a Type I error rate equal to $a_Cα$. The decision rules for $H_{01}$ are then calibrated so that the overall familywise Type I error rate is equal to $α$.


\end{itemize}

\subsection*{Advanced Parameters}

\begin{itemize}

\item Delta (δ): This parameter defines the curvature of the efficacy and futility boundaries, which are all proportional to $\{(\sum_{k'=1}^{K} n_{k'})/n_k\}^{-δ}$.

\item Number of Iterations for simulation: Z-statistics are simulated generate the power, expected sample size, expected trial duration, and expected number of overrunning patients. Generally, about 10,000 simulations are needed for reliable results. It is our experience that a simulation with 10,000 iterations takes about 15 seconds on a modern personal computer.

\item Time limit for simulation, in seconds: If the simulation time exceeds this threshold, calculations will stop and the user will get an error message saying that the application has “reached CPU time limit”. To remove the error, either the number of iterations can be reduced, or the time limit for simulation can be extended. \interAdapt does not allow for this time limit to exceed 90 seconds.

\item Total number of stages (K): The total number of stages for all three designs. 

\item Participants enrolled per year from combined population: The number of patients that can be recruited per year in the combined population. This affects the estimated duration of the trials. The recruitment rates for  subpopulations $1$ and $2$ are equal to the combined population recruitment rate multiplied by $π_1$ and $π_2$ respectively. %Right??!!! This is new!!!???

\item Delay time from enrollment to primary outcome in years: The between when an individual is enrolled in the study, and when their final outcome is measured. This value affects the expected number of overruning patients, and the expected duration of the trial.

%!!!??? are these right? See email to Michael.
\item Lower bound for treatment effect in sub-population $2$: \interAdapt simulates performance metrics under a range of treatment effect sizes for subpopulation $2$. This sets the lower bound for this range.

\item Upper bound for treatment effect in sub-population $2$: \interAdapt simulates performance metrics under a range of treatment effect sizes for subpopulation $2$. This sets the upper bound for this range.
%!?!?! Need to enforce that lower bound is less than upper bound!!??

\item Last stage sub-population $1$ is enrolled under an adaptive design ($k^*$): In the adaptive design, we don’t enroll any patients from subpopulation $2$ after stage $k^*$. %!!! Adjust labels for this in boxTable and code!!!!???

\item Per stage sample size for standard group sequential design enrolling combined pop ($n_{SC}$): The number of patients enrolled in each stage for $SC$.

\item Per stage sample size for standard group sequential design enrolling only subpop. 1 ($n_{SS}$): The number of patients enrolled in each stage for $SS$.

\item $H_{0C}$ futility boundary proportionality constant for the adaptive design ($f_{AD,C}$): This is used to calculate the futility boundary for $H_{0C}$ in the adaptive design, which is set to $f_{AD,C}\{(\sum_{k'=1}^{K} n_{k'})/n_k\}^{-δ}$ in stage $k$.

%!!!??? Need to make sure the labels in boxTable say standard instead of fixed for these last three values.
\item $H_{01}$ futility boundary proportionality constant for the adaptive design ($f_{AD,S}$):  This is used to calculate the futility boundary for $H_{01}$ in the adaptive design, which is set to $f_{AD,S}\{(\sum_{k'=1}^{K} n_{k'})/n_k\}^{-δ}$ in stage $k$.

\item $H_{0C}$ futility boundary proportionality constant for the standard design ($f_{SC}$): This is used to calculate the futility boundary for $H_{0C}$ in $SC$, which is set to $f_{SC}\{(\sum_{k'=1}^{K} n_{k'})/n_k\}^{-δ}$ in stage $k$.

\item $H_{01}$ futility boundary proportionality constant for the standard design ($f_{SS}$):  This is used to calculate the futility boundary for $H_{01}$ in $SS$, which is set to $f_{SS}\{(\sum_{k'=1}^{K} n_{k'})/n_k\}^{-δ}$ in stage $k$.


\end{itemize}

\subsection*{Outputs}

The output panel of the user interface is split into three sections, ``About interAdapt", ``Designs'' output and ``Performance'' output. The ``About interAdapt'' section gives a brief introduction to the software, and a link to the full software documentation. The Designs output gives a road plan for how to conduct each of the three trials: $FA$, $AD$ and $SC$. This includes the efficacy boundaries; user specified non-binding futility boundaries, and number of patients to recruit by the end of each stage. Performance output compares the three designs in terms of their power, expected sample size, expected duration, and expected number of overrunning patients. 
The radio buttons at the top of the output section can be used to switch between these three sections. 


%read up to here 2013-11-27 midnight/early morning.!!!
\subsection*{Designs Output}

The designs output gives information on how to conduct each of the three trials. Tabs at the top of the page can be used to navigate between the results for each design. Each of the first three tabs each correspond with one of the designs, and the fourth tab shows all three designs side by side. 

In the “Adaptive” tab, the table at the bottom of the page shows the required number of patients that must be recruited by the end of each stage.For each stage $k$, the table also gives efficacy boundaries for $Z_{1,k}$ and $Z_{C,k}$, and futility boundaries for $Z_{1,k}$ and $Z_{2,k}$. Because we always stop enrolling subpopulation $2$ after stage $k^*$, futility boundaries for $Z_{2,k}$ in stage $k^*$ and later stages are not given. For the same reason, efficacy boundaries for $Z_{C,k}$ are not given for stages $k>k^*$. A plot at the top of the page shows these efficacy and futility boundaries for $Z_{C,k}$, $Z_{1,k}$ and $Z_{2,k}$ over all stages of the trial.

The two tabs for the standard designs have a comprable layout. Note that the efficacy boundaries for $SS$ and $SC$ are identical. This is because the efficacy boundary depends only on the null distribution of z-statistics, which unaffected by the choice of study population. The two standard trials each pull from only one subpopulation, so their efficacy boundaries are the same.

The final tab combines the tables from the first three tabs, and omits plots of the decision boundaries.


\subsection*{Performance Output -- Layout \& Interpretation}

\interAdapt shows performance of each of the three designs in terms of four metrics: power, expected sample size, expected duration, and expected number of overrunning patients. These metrics all depend, among other things, on the true treatment effect in each subpopulation. A treatment effect for subpopulation $1$ can be specified in the Basic Parameters section, and a range of values for the treatment effect in subpopulation $2$ can be specified in the Advanced Parameters section. %!!!???!!!??? See earlier note in parameters section.
 \interAdapt will calculate performance metrics for the specified range of treatment effects, and generate charts of each metric plotted against the underlying treatment effect in subpopulation $2$. These four plots can be accessed via the tabs at the top of the page. The table at bottom of the output section shows all four metrics side by side, with each column of the table denoting a different treatment effect in subpopulation $2$.%!!!???!!!??? See earlier note in parameters section.

When the true treatment is very strong, trials will tend to be able to detect the treatment effect more easily, and will be more likely to stop early for efficacy. This translates to an overall increase in power, a decrease in expected sample size, a decrease in expected trial duration, and an increase in the expected number of overrunning patients. The increase in overruning patients is due to the fact that it is precisely the process of stopping early that generates overrunning patients. Conversely, if the true underlying treatment effect is significantly harmful, the trials will be more likely to stop early for futility. This leads to trials with smaller expected sample sizes, shorter expected durations, and larger expected numbers of overrunning patients. Trials will tend to last the longest when the treatment effect is positive, but not overwhelmingly strong. These patterns are reflected in the plots shown by \textsf{interAdapt}.

The power plot shows the power of $AD$ to reject $H_{0C}$, to reject $H_{01}$, and to reject at least one of $H_{0C}$ or $H_{01}$. As the standard design $SC$ only tests $H_{0C}$, \interAdapt only shows it’s power to reject $H_{0C}$. Likewise, \interAdapt only shows the power of $SS$ to reject $H_{01}$. Note that the power of $SC$ and $AD$ to reject $H_{0C}$ both increase as the treatment effect for subpopulation $2$ increases. The power of $AD$ to reject $H_{01}$ decreases as the treatment effect in subpopulation $2$ increases, but this is only because $AD$ does not bother to test $H_{01}$ after a treatment effect in the combined population is discovered.

In general, power of a trial can be increased by increasing the per-stage sample size ($n_1^*$, $n_k^*$, $n_{SS}$ and $n_{SC}$), increasing the number of stages ($K$), lowering the futility boundaries ($f_{SC}$, $f_{SS}$, $f_{AD,C}$, or $f_{AD,S}$), or relaxing the required Type I error rate (α).

The power of $SS$ is constant with respect to the true treatment effect in subpopulation $2$, as we'd expect since $SS$ does not take any data from subpopulation $2$. The expected sample size, expected duration, and the expected number of overrunning patients for $SS$ are also constant with respect to the true treatment effect in subpopulation $2$.

In the plot of expected sample size for each design, we see that trials tend to need to recruit more patients when the treatment effect is weak. For designs testing for an effect in the combined population, this means that the expected sample size will be highest when the weighted average treatment effect across subpopulations is weak. If the treatment effect is significantly positive in subpopulation $1$, the highest possible expected sample size may come at a negative value for the true treatment effect in subpopulation $2$. In general, lowering $K$ or $k^*$, increasing the futility bounds ($f_{SC}$, $f_{SS}$, $f_{AD,C}$, or $f_{AD,S}$), or relaxing the required Type I error rate ($α$), can all decrease the expected sample size.

The plot of expected trial duration for each design shows patterns very similar to those in the plot of expected sample size. A trial's duration is defined as the time until the last patient's outcome is measured. Like expected sample size, the expected duration can be decreased by lowering $K$ or $k^*$, increasing the futility bounds ($f_{SC}$, $f_{SS}$, $f_{AD,C}$, or $f_{AD,S}$), or relaxing $α$. Increasing the recruitment rate, or decreasing the delay time to outcome measurement, can also shorten the expected duration of a trial.

The plot of the expected number of overrunning patients will generally show a minimum when the treatment effect is just on the cusp of significance, as this will require trials to gather more data. Whenever a trial gathers as much data as possible and reaches stage $K$,  there will be no overrunning patients. When applicable, decreasing $K$ or $k^*$, decreasing the futility bounds ($f_{SC}$, $f_{SS}$, $f_{AD,C}$, or $f_{AD,S}$), decreasing the delay time to outcome measurement, or lowering the recruitment rate can all decrease the expected number of overrunning patients.



\section*{6. Example of Entering Input and Interpreting Output}

The default inputs to \interAdapt come from the motivating example of the MISTIE Phase III trial. This section presents a summary of this trial, and of the design goals of the investigators, as described in (Rosenblum et al. 2013)\cite{Rosenblum2013AdaptMISTIE}. The MISTIE trial studied a new surgical treatment for stroke, and measured patient's outcomes by their disability score on the modified Rankin Scale (mRS) 180 days after enrollment. A successful outcome was defined as a mRS score less than or equal to 3. %Still measured 180 days after enrollment, or sooner???!!! New draft of paper seems to have it different than before, and says you "focus on the case of outcomes measured soon after enrollment"

The Phase II trial for the MISTIE treatment had only enrolled patients with with little or no intraventricular hemorrhage (IVH). More specifically, patients had been categorized as ``small IVH'' if their IVH volume was less than 10ml, and did not require a catheter for intracranial pressure monitoring. Otherwise, pateints were classified as ``large IVH.'' The Phase II trial only recruited small IVH patients, and yeiled a treatment effect estimate of 12.1\% [95\% CI: (-2.7\%, 26.9\%)]. The investigators thought that the treatment could also be effective in large IVH pateints, but no data had yet been collected to test this. Thus, we refer to the subpopulation of small IVH patients as subpopulation $1$, as there was more prior evidence of treatment efficacy in this subpopulation.

 The study designers were concerned with the calibrating power and alpha level of the Phase III trial under the following three scenarios:


\begin{description}
\item  (a) The average treatment effect is $12.5\%$ for both small and large IVH pateints;
\item  (b) The average treatment effect is $12.5\%$ for small IVH patients, and zero large IVH patients;
\item  (c) The treatment effect is zero both subpopulations. 
\end{description}

In the context of these scenarios, the study coordinators had three goals:

\begin{description}
\item  (i) At least 80\% power for testing $H_{0C}$ in scenario (a);
\item  (ii) At least 80\% power for testing $H_{01}$ in scenario (b);
\item  (iii) A familywise Type I error rate (α) of .025.
\end{description}

Prior research by (Hanley 2012)\cite{Hanley2012} indicated that the proportion of patients with small IVH ($π_1$) was .33, that the probability of a positive outcome under control was .25 for small IVH patients ($p_{1c}$), and that the probability of a positive outcome under control was .2 for large IVH patients ($p_{2c}$).  If the true treatment effect in subpopulation $1$ was 12.5\% then the probability of a positive outcome under treatment for patients in subpopulation $1$ ($p_{1t}$) would be approximately 12.5\%+25\%=37.5\%.

Since the adaptive design $AD$ tests $H_{0C}$ as well as $H_{01}$, it must achieve all three goals (i)-(iii). The standard design $SC$ need only achieve (i) and (iii), and the standard design $SS$ need only achieve (ii) and (iii). Recall that \interAdapt allows the user to specify a range of treatment values for subpopulation $2$, and will display the power of the trial designs across this range. By default, \interAdapt sets the range of values for the treatment affect in subpopulation $2$ to [-.2, .2], letting the user see the power of all three designs under scenarios (a) and (b). %!!!!????? adjust this!!??
 
The remaining default input parameters come from the analysis section of (Rosenblum et al. 2013)\cite{Rosenblum2013AdaptMISTIE}. Here, the authors first fixed $K=5$ and $δ=-.5$, and then searched for values of the remaining parameters that minimize the average expected sample size over scenarios (a)-(c) for the adaptive design, while still achieving goals (i)-(iii). They found a minimum average expected sample size at $k^*=4$, $n_1^*=150$, $n_k^*=311$, and $f_{AD,C}=f_{AD,S}=0$. %!!!???? Is this even a relevant way to phrase it anymore!!!???

Now we turn to the output of \interAdapt that results from the default parameters, and show that each of the three designs achieves its relevant goals. In the power plot, we see that $AD$ has 80\% power to reject $H_{0C}$ in scenario (a), and 80\% power to reject $H_{01}$ in scenario (b). $SC$ has 80\% power to reject $H_{0C}$ in scenario (a), and $SS$ has 80\% power to reject $H_{01}$ in scenario (b). Although it is not shown, we know that the familywise Type I error rate is less than .025, as this was specified as an input to \textsf{interAdapt}. %It looks like the power is actually a little too low!!!!!??????




\section*{Summary}

We described the \interAdapt application for designing and simulating trials with adaptive enrollment criteria. We provided an overview of the theoretical problem the application addresses, and gave an explanation of the application’s inputs and outputs.

Current limitations of the software include that the outcome is assumed to be binary.



\section*{Acknowledgements}




\bibliographystyle{plain}
\bibliography{interAdapt}


\end{document}